% LaTeX file for resume 
% This file uses the resume document class (res.cls)

\documentclass[margin]{res}
%\usepackage{helvetica} % uses helvetica postscript font (download helvetica.sty)
%\usepackage{newcent}   % uses new century schoolbook postscript font  
\topmargin=-0.5in  % start text higher on the page
\setlength{\textheight}{12in} % increase text height to fit resume on 1 page
\usepackage[utf8]{inputenc}
\usepackage{url}
\usepackage{graphicx}
\usepackage{wrapfig}
\begin{document}
\name{FLÁVIO CRUZ}

\address{Albisstrasse, 8 \\ 8134 Adliswil, ZH, Switzerland}
%\address{Rua 1º de Maio, nº 48 \\   3515-560 Bodiosa, Portugal \\   (+351) 936 229 412 \\ flaviocruz@gmail.com \\ \url{http://cs.cmu.edu/~fmfernan}}

\begin{resume}

\section{EDUCATION}

                Carnegie Mellon University and University of Porto \\
                PhD in Computer Science (2011-2016) \\
                Research on a linear logic programming language for parallel
                programming
                %Advised by Seth Goldstein (\url{seth@cs.cmu.edu}), Frank Pfenning (\url{fp@cs.cmu.edu}) and Ricardo Rocha (\url{ricroc@fc.up.pt})\\

Faculdade de Engenharia da Universidade do Porto \\
                Integrated Masters in Informatics and Computing Engineering (2005-2010) \\

\section{INTERESTS}

                 \begin{itemize}
                    \item Parallel and Distributed Systems
                    \item Compilers and Implementation of Programming Languages
                 \end{itemize}
                  
\section{SKILLS} 
\normalsize{\section{Programming Languages}}
                 \begin{itemize}
                 \item C, C++, Java, Lisp, Python
                 \end{itemize}
                 
\section{EXPERIENCE}
            
                  \begin{tabular}{p{3in} r} % one column is 3 inches wide
                    Google Zurich &  From October 2015
                   \end{tabular}
                    \begin{itemize}
                      \item[] Software engineer working on AdWords.
                    \end{itemize}
                  \begin{tabular}{p{3in} r} % one column is 3 inches wide
                    Carnegie Mellon University &  From September 2014 to December 2014
                   \end{tabular}
                    \begin{itemize}					        
                     \item[] Graduate Teaching Assistant for the 15-411 Compiler Design course at CMU. Graded homework assignments and helped students develop their compilers during weekly held office hours.
                    \end{itemize}

                  \begin{tabular}{p{3in} r} % one column is 3 inches wide
                    Google Zurich &  From June 2014 to September 2014
                   \end{tabular}
                    \begin{itemize}
                      \item[] Worked as a software engineering intern on the GRR project, a distributed incident response framework.
                    \end{itemize}

                  \begin{tabular}{p{3in} r} % one column is 3 inches wide
                    Faculdade de Ciencias UP &  From September 2013 to January 2014
                   \end{tabular}
                    \begin{itemize}					        
                     \item[] Graduate Teaching Assistant for the Parallel and Distributed Programming course (Masters in Computer Science). Designed the student assignments using MPI and Pthreads and helped students during lab classes.
                    \end{itemize}

                  \begin{tabular}{p{3in} r}
                    Summer Internship at CMU / Intel Research Labs & From July 2010 to October 2010
                  \end{tabular}
                  \begin{itemize}
                    \item[] Extending the Meld language and runtime to program parallel systems. Work supervised by Seth Copen Goldstein.
                  \end{itemize}
                  \clearpage
                  
                  \begin{tabular}{p{3in} r}
                    Scholarship at CRACS/FCUP & From February 2010 to June 2010
                  \end{tabular}
                  \begin{itemize}
                    \item[] Implementation of a tabling engine based on \textit{call-subsumption} semantics for Yap Prolog. Supervised by Ricardo Rocha.
                  \end{itemize}
                  

                  \begin{tabular}{p{3in} r} % one column is 3 inches wide
                    Faculdade de Engenharia UP &  From October 2009 to December 2009
                   \end{tabular}
                    \begin{itemize}					        
                     \item[] Undergraduate Teaching Assistant for the Graphical Applications Laboratory course (Integrated Masters in Informatics and Computer Engineering). Supported students during the development of various OpenGL/C++ projects.
                    \end{itemize}
                    
                 \begin{tabular}{p{3in} r} % one column is 3 inches wide
                  Faculdade de Engenharia UP &  From March 2009 to June 2009
                 \end{tabular}
                  \begin{itemize}					        
                   \item[] Undergraduate Teaching Assistant for the Programming 2 course (Masters in Electrical and Computers Engineering). Supported students in the development of a small project written in C++, with a focus on object oriented programming.
                  \end{itemize}


		 \begin{tabular}{p{3in} r}
                  Scholarship at IBMC &  From November 2008 to November 2009 
                 \end{tabular}
		  \begin{itemize}
                   \item[] Design and implementation of a flexible biological database. Supervised by Nuno Fonseca.
                  \end{itemize}
                  

                  \begin{tabular}{p{3in} r}
                  Google Summer of Code 2008 & Summer of 2008
                  \end{tabular}	
                   \begin{itemize} % \item[] prevents a bullet from appearing
                    \item[]  Design and implementation of a filesystem library written in Common Lisp for easy development of filesystem servers. Library developed for the GNU Hurd project. 
		   \end{itemize} 
		   
\section{PUBLICATIONS}
   \begin{itemize}

      \item Declarative Coordination of Graph-Based Parallel Programs. Flavio
            Cruz, Seth Copen Goldstein and Ricardo Rocha.  21st ACM SIGPLAN
            Symposium on Principles and Practice of Parallel Programming (PPoPP
            2016).

      \item Thread-Aware Logic Programming For Data-Driven Parallel Programs.
            Flavio Cruz, Richardo Rocha and Seth Copen Goldstein. 31th
            International Conference on Logic Programming (ICLP 2015).

      \item A Scalable File Based Data Store for Forensic Analysis. Flavio Cruz,
         Andreas Moser, Michael Cohen. Journal Digital Investigation. Dublin,
         Ireland.

      \item Design and Implementation of a Multithreaded Virtual Machine for
         Executing Linear Logic Programs. Flavio Cruz, Ricardo Rocha and Seth
         Copen Goldstein. 16th International Symposium on Principles and
         Practice of Declarative Programming (PPDP 2014), ACM Press. Canterbury,
         UK.

      \item A Parallel Virtual Machine for Executing Forward-Chaining Linear Logic Programs. Flavio Cruz, Ricardo Rocha and Seth Copen Goldstein. Proceedings of the International Joint Workshop on Implementation of Constraint and Logic Programming Systems and Logic-based Methods in Programming Environments (CICLOPS-WLPE 2014). Vienna, Austria.
      \item A Linear Logic Programming Language for Concurrent Programming over Graph Structures. Flavio Cruz, Ricardo Rocha, Seth Goldstein and Frank Pfenning. Journal of Theory and Practice of Logic Programming, 30th International Conference on Logic Programming (ICLP 2014), Special Issue. Vienna, Austria. (\textbf{BEST PAPER AWARD})
      \item Bottom-Up Logic Programming for Multicores. Flavio Cruz, Michael P. Ashley-Rollman, Seth Copen Goldstein, Ricardo Rocha and Frank Pfenning. Declarative Aspects and Applications of Multicore Programming (DAMP 2012). Philadelphia, PA, USA.
      \item Single Time-Stamped Tries for Retroactive Call Subsumption. Flávio Cruz and Ricardo Rocha. Proceedings of the 11th Colloquium on Implementation of Constraint and LOgic Programming Systems (CICLOPS 2011). Lexington, Kentucky, USA.
\end{itemize}
\pagebreak
\section{PUBLICATIONS}
\begin{itemize}
      \item Efficient Instance Retrieval of Subgoals for Subsumptive Tabled Evaluation of Logic Programs. Flávio Cruz and Ricardo Rocha. 27th International Conference on Logic Programming (ICLP 2011). Lexington, Kentucky, USA.
      %\item BioSeD: A Biological Sequences Database. Flávio Cruz, Cristina Vieira, Jorge Vieira, Nuno A. Fonseca.
      % Oxford Journal of Bioinformatics. (In Progress).
      \item BioSeD - Biological Sequences Database. Flávio Cruz. Technical Report. Institute for Molecular and Cell Biology. 2009.
      \item Call Subsumption Mechanisms for Tabled Logic Programs. Flávio Cruz. MSc in Informatics and Computing Engineering, Department of Informatics Engineering, Faculty of Engineering, University of Porto. Portugal, June 2010.
      \item Retroactive Subsumption-Based Tabled Evaluation of Logic Programs.
      Flávio Cruz and Ricardo Rocha. 12th European Conference on Logics in Artificial Intelligence (JELIA 2010), Springer-Verlag. Helsinki, Finland, September 2010.
      \item Efficient Retrieval of Subsumed Subgoals in Tabled Logic Programs.
      Flávio Cruz and Ricardo Rocha. 4th International Conference on Compilers, Programming Languages, Related Technologies and Applications (CoRTA 2010). Braga, Portugal, September 2010.
   \end{itemize}


\section{SCHOOLS}
   \begin{itemize}
      \item WACS: Winter Advanced Computing Seminars 2011, Braga, Portugal.
      \item ISCL 2011: Third International ALP/GULP Spring School on Computational Logic, Bertinoro, Italy.
   \end{itemize}

\section{LANGUAGES}
      \begin{itemize}
        \item Portuguese: native speaker.
        \item English: experienced in reading, writing and speaking.
      \end{itemize}

\section{OTHERS} Merit Scholarship "Ribas de Sousa 2002/2003", awarded by Escola Secundária Emídio Navarro to the best student on the 10º grade. \\
                  “Prémio Incentivo”, awarded by University of Porto to the best students finishing the first year of their degrees. \\
                  Participation on CPUP 2007 (Programming Contest) with Miguel Oliveira and João Azevedo. 2nd place among 8 teams. \\
                  

\end{resume}

\end{document}
