% LaTeX file for resume 
% This file uses the resume document class (res.cls)

\documentclass[margin]{res}
%\usepackage{helvetica} % uses helvetica postscript font (download helvetica.sty)
%\usepackage{newcent}   % uses new century schoolbook postscript font  
\topmargin=-0.5in  % start text higher on the page
\setlength{\textheight}{12in} % increase text height to fit resume on 1 page
\usepackage[utf8]{inputenc}
\usepackage{url}
\usepackage{graphicx}
\usepackage{wrapfig}
\begin{document}
\name{FLÁVIO CRUZ}

\address{Rua 1º de Maio, nº 48 \\   3515-560 Bodiosa, Portugal \\   (+351) 936 229 412 \\ flaviocruz@gmail.com \\ \url{http://cs.cmu.edu/~fmfernan}}

\begin{resume}

\section{EDUCATION}       Faculdade de Engenharia da Universidade do Porto \\
                Integrated Masters in Informatics and Computing Engineering (2005-2010) \\
                Grade average of 18 out of 20 (top 2\%)
                
                Carnegie Mellon University and University of Porto \\
                PhD in Computer Science (2011-2015) \\
                Research on a linear logic programming language for parallel programming \\
                Advised by Seth Goldstein (\url{seth@cs.cmu.edu}), Frank Pfenning (\url{fp@cs.cmu.edu}) and Ricardo Rocha (\url{ricroc@fc.up.pt})\\

\section{INTERESTS}

                 \begin{itemize}
                    \item Implementation of Programming Languages
                    \item Parallel and Distributed Programming
                    \item Logic and Proof Theory
                 \end{itemize}
                  
\section{SKILLS} 
\normalsize{\section{Programming Languages}}
                 \begin{itemize}
                 \item C, C++, Common Lisp, Haskell, Coq
                 \item Prolog, SQL, Javascript, Python
                 \end{itemize}
                 
\normalsize{\section{Technologies}} 
                 \begin{itemize}
                  \item Subversion, git, MySQL, Vim, TextMate
                  \item Linux, Windows, MacOSX, GNU Hurd
                  \end{itemize}
                  
\normalsize{\section{Others}}
              \begin{itemize}
                \item Web Development: HTML, CSS, jQuery, JSON, XML
              \end{itemize}

\section{EXPERIENCE}
            
                  \begin{tabular}{p{3in} r}
                  Google Inc. - Google Summer of Code 2008 & Summer of 2008
                  \end{tabular}	
                   \begin{itemize} % \item[] prevents a bullet from appearing
                    \item[]  Design and implementation of a filesystem library written in Common Lisp for easy development of filesystem servers. Library developed for the GNU Hurd project. 
		   \end{itemize} 
		   
		 \begin{tabular}{p{3in} r}
                  Scholarship at IBMC &  From November 2008 to November 2009 
                 \end{tabular}
		  \begin{itemize}
                   \item[] Design and implementation of a flexible biological database. Supervised by Nuno Fonseca.
                  \end{itemize}
                  
                 \begin{tabular}{p{3in} r} % one column is 3 inches wide
                  Faculdade de Engenharia UP &  From March 2009 to June 2009
                 \end{tabular}
                  \begin{itemize}					        
                   \item[] Undergraduate Teaching Assistant for the Programming 2 course (Masters in Electrical and Computers Engineering). Supported students in the development of a small project written in C++, with a focus on object oriented programming.
                  \end{itemize}
                  
                  \begin{tabular}{p{3in} r} % one column is 3 inches wide
                    Faculdade de Engenharia UP &  From October 2009 to December 2009
                   \end{tabular}
                    \begin{itemize}					        
                     \item[] Undergraduate Teaching Assistant for the Graphical Applications Laboratory course (Integrated Masters in Informatics and Computer Engineering). Supported students during the development of various OpenGL/C++ projects.
                    \end{itemize}
                    
                  \begin{tabular}{p{3in} r}
                    Scholarship at CRACS/FCUP & From February 2010 to June 2010
                  \end{tabular}
                  \begin{itemize}
                    \item[] Implementation of a tabling engine based on \textit{call-subsumption} semantics for Yap Prolog. Supervised by Ricardo Rocha.
                  \end{itemize}
                  
                  \begin{tabular}{p{3in} r}
                    Summer Internship at CMU / Intel Research Labs & From July 2010 to October 2010
                  \end{tabular}
                  \begin{itemize}
                    \item[] Extending the Meld language and runtime to program parallel systems. Work supervised by Seth Copen Goldstein.
                  \end{itemize}
                  \clearpage
                  
                  \begin{tabular}{p{3in} r} % one column is 3 inches wide
                    Faculdade de Ciencias UP &  From September 2013 to January 2014
                   \end{tabular}
                    \begin{itemize}					        
                     \item[] Graduate Teaching Assistant for the Parallel and Distributed Programming course (Masters in Computer Science). Designed the student assignments using MPI and Pthreads and helped students during lab classes.
                    \end{itemize}

\section{OTHERS} Merit Scholarship "Ribas de Sousa 2002/2003", awarded by Escola Secundária Emídio Navarro to the best student on the 10º grade. \\
                  “Prémio Incentivo”, awarded by University of Porto to the best students finishing the first year of their degrees. \\
                  Participation on CPUP 2007 (Programming Contest) with Miguel Oliveira and João Azevedo. 2nd place among 8 teams. \\
                  
\section{PUBLICATIONS}
   \begin{itemize}
      \item Bottom-Up Logic Programming for Multicores. Flavio Cruz, Michael P. Ashley-Rollman, Seth Copen Goldstein, Ricardo Rocha and Frank Pfenning. Declarative Aspects and Applications of Multicore Programming (DAMP 2012). Philadelphia, PA, USA.
      \item Single Time-Stamped Tries for Retroactive Call Subsumption. Flávio Cruz and Ricardo Rocha. Proceedings of the 11th Colloquium on Implementation of Constraint and LOgic Programming Systems (CICLOPS 2011). Lexington, Kentucky, USA.
      \item Efficient Instance Retrieval of Subgoals for Subsumptive Tabled Evaluation of Logic Programs. Flávio Cruz and Ricardo Rocha. 27th International Conference on Logic Programming (ICLP 2011). Lexington, Kentucky, USA.
      %\item BioSeD: A Biological Sequences Database. Flávio Cruz, Cristina Vieira, Jorge Vieira, Nuno A. Fonseca.
      % Oxford Journal of Bioinformatics. (In Progress).
      \item BioSeD - Biological Sequences Database. Flávio Cruz. Technical Report. Institute for Molecular and Cell Biology. 2009.
      \item Call Subsumption Mechanisms for Tabled Logic Programs. Flávio Cruz. MSc in Informatics and Computing Engineering, Department of Informatics Engineering, Faculty of Engineering, University of Porto. Portugal, June 2010.
      \item Retroactive Subsumption-Based Tabled Evaluation of Logic Programs.
      Flávio Cruz and Ricardo Rocha. 12th European Conference on Logics in Artificial Intelligence (JELIA 2010), Springer-Verlag. Helsinki, Finland, September 2010.
      \item Efficient Retrieval of Subsumed Subgoals in Tabled Logic Programs.
      Flávio Cruz and Ricardo Rocha. 4th International Conference on Compilers, Programming Languages, Related Technologies and Applications (CoRTA 2010). Braga, Portugal, September 2010.
   \end{itemize}
   
\section{SCHOOLS}
   \begin{itemize}
      \item WACS: Winter Advanced Computing Seminars 2011, Braga, Portugal.
      \item ISCL 2011: Third International ALP/GULP Spring School on Computational Logic, Bertinoro, Italy.
   \end{itemize}

\section{LANGUAGES}
      \begin{itemize}
        \item Portuguese: native speaker.
        \item English: experienced in reading, writing and speaking.
      \end{itemize}

\section{PERSONAL INTERESTS}

   \begin{itemize}
      \item Electronic Music Production
      \item Powerlifting
   \end{itemize}

\end{resume}

\end{document}