% LaTeX file for resume 
% This file uses the resume document class (res.cls)

\documentclass[margin]{res}
%\usepackage{helvetica} % uses helvetica postscript font (download helvetica.sty)
%\usepackage{newcent}   % uses new century schoolbook postscript font  
\topmargin=-0.5in  % start text higher on the page
\setlength{\textheight}{12in} % increase text height to fit resume on 1 page
\usepackage[utf8]{inputenc}
\usepackage{url}
\begin{document}
\name{FLÁVIO CRUZ}

\address{ Rua 1º de Maio, nº 48 \\   3515-560 Bodiosa \\   (+351) 934 908 448 \\ flaviocruz@gmail.com \\ \url{http://flaviocruz.net}}
                           
                        
\begin{resume}

\section{EDUCAÇÃO}       Faculdade de Engenharia da Universidade do Porto \\
                Mestrado Integrado em Engenharia Informática e Computação, a finalizar em 2010 \\
                Média 17.68
 
                \begin{ncolumn}{1}
                {\bf Principais Disciplinas} \\
                Programação (19)         \\
                Algoritmos e Estruturas de Dados (19)  \\
                Implementação de Linguagens (na FCUP, 19) \\
                Compiladores (18)      \\
                Inteligência Artificial (19)
		\end{ncolumn}

\section{INTERESSES}

\normalsize{\section{Professionais}} 
                 \begin{itemize}
                  \item Implementação de Linguagens de Programação
                  \item Programação Funcional e Lógica
                  \item Inteligência Artificial
                  \end{itemize}
\section{SKILLS} 
\normalsize{\section{Linguagens de Programação}}
                 \begin{itemize}
                 \item C, C++, Common Lisp, PHP
                 \item Haskell, Prolog, SQL, Javascript, Python
                 \end{itemize}
                 
\normalsize{\section{Tecnologias}} 
                 \begin{itemize}
                  \item Subversion, git, PostgreSQL, MySQL, Vim, TextMate
                  \item Linux, Windows, MacOSX, GNU Hurd
                  \end{itemize}
                  
\normalsize{\section{Outros}} 
              \begin{itemize}
                \item Desenvolvimento Web: HTML, CSS, jQuery, JSON, XML
              \end{itemize}
 
\section{EXPERIÊNCIA}
            
                  \begin{tabular}{p{3in} r}
                  Google Inc. - Google Summer of Code 2008 & Verão de 2008
                  \end{tabular}	
                   \begin{itemize} % \item[] prevents a bullet from appearing
                    \item[]  Desenho e implementação de uma biblioteca, escrita em Common Lisp, para criação de servidores
                     de sistema de ficheiros, para o projecto de software livre GNU Hurd 
		   \end{itemize} 
		   
		 \begin{tabular}{p{3in} r}
                  Bolsa de Integração na Investigação no IBMC &  Novembro de 2008 a Novembro de 2009 
                 \end{tabular}
		  \begin{itemize}
                   \item[] Desenho e implementação de uma base de dados biológica flexível. Orientada pelo Dr. Nuno Fonseca.
                  \end{itemize}
                  
                 \begin{tabular}{p{3in} r} % one column is 3 inches wide
                  Faculdade de Engenharia UP &  Março de 2009 a Junho de 2009
                 \end{tabular}
                  \begin{itemize}					        
                   \item[] Monitor da disciplina de Programação II do MIEEC.
                  \end{itemize}
                  
\section{OUTROS} Prémio Ribas de Sousa 2002/2003 – Prémio oferecido ao melhor aluno do 10º ano na Escola
                  Secundária Emídio Navarro. \\
                  “Prémio Incentivo” da Universidade do Porto, entregue aos melhores alunos da universidade
                  no 1º ano do plano de estudos. \\
                  Participação no CPUP 2007 juntamente com Miguel Oliveira e João Azevedo. 2º lugar entre 8 equipas. \\
                  
\section{LÍNGUAS}
      \begin{itemize}
        \item Português: Língua Materna
        \item Inglês: Experiente na leitura e escrita. Independente na conversação oral.
      \end{itemize}
 
\end{resume} 
\end{document}